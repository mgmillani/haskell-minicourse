\documentclass{beamer}
\usepackage[T1]{fontenc}
\usepackage[utf8]{inputenc}
\usepackage[brazilian]{babel}
\usepackage{listingsutf8}
\usepackage{color}
\usepackage{theoremref}
\usepackage{verbatim,amssymb,latexsym,amsmath,mathrsfs}

\usecolortheme{rose}
\usetheme{Hannover}

\hypersetup
{
  colorlinks=true,
  urlcolor=blue,
  citecolor=blue,
  linkcolor=blue
}

\definecolor{darkGreen}{rgb}{0,0.6,0}

\lstset{
	extendedchars=true,
	inputencoding=utf8,
	basicstyle=\small,
	keywordstyle=\color{red}\bfseries,
	identifierstyle=\color{darkGreen},
	commentstyle=\color{gray}\textit,
	stringstyle=\color{magenta},
	showstringspaces=false,
	tabsize=1,
	language=Haskell,
	numbers=left,         % Números na margem da esquerda
	numberstyle=\tiny,    % Números pequenos
	stepnumber=1,         % De 1 em 1
	numbersep=5pt         % Separados por 5pt
}

\lstset
{
	literate={ê}{{\^e}}1 {ç}{{\c{c}}}1 {ã}{{\~a}}1 {â}{{\^a}}1 {õ}{{\~o}}1 {í}{{\'i}}1 {é}{{\'e}}1 {á}{{\'a}}1 {ó}{{\'o}}1 {ú}{{\'u}}1
}

\title{Mini-Curso de Haskell - SEMAC 2013/1}
\author[M. Millani]{Marcelo Garlet Millani}
\institute[II/UFRGS]{Universidade Federal do Rio Grande do Sul \\ Instituto de Informática \\ Grupo PET Computação}

\begin{document}

	\begin{frame}
		\maketitle
	\end{frame}

\section{Sobre a Linguagem}
	\begin{frame}{O que é Haskell}
		\begin{itemize}
		 \item Linguagem puramente funcional (não permite efeitos colaterais)
		 \item Possui um sistema de mônadas para isolar partes impuras do código
		 \item Usa avaliação preguiçosa (só calcula uma expressão se ela for necessária)
		\end{itemize}
	\end{frame}
	
	\begin{frame}{História}
	 
	 \begin{itemize}
	  \item Proposta na conferência FPCA '87 (Functional Programming Languages and Computer Architecture)
	  \item Havia diversas linguagens funcionais não-estritas, todas similares
	  \item Foi criada com o objetivo de propor uma linguagem funcional padrão
	  \item Nome em homenagem a Haskell Curry
	 \end{itemize}
	\end{frame}
	
	\begin{frame}{História}
	 A linguagem proposta deveria satisfazer o seguinte:
	 \begin{itemize}
	  \item Deveria ser adequada para o ensino, pesquisa e aplicações (incluindo sistemas grandes)
	  \item Deveria ser descrita completamente através da publicação de sintaxe e semânticas formais
	  \item Deveria estar disponível livremente, de forma que qualquer um pudesse implementar e distribuir a linguagem
	  \item Deveria ser baseada em ideias amplamente aceitas
	  \item Deveria reduzir diversidade desnecessária em linguagens funcionais
	 \end{itemize}
	\end{frame}
	
\section{Ferramentas}

	\begin{frame}{Interpretadores e Compiladores}
	 
	 \begin{description}
	  \item [Hugs] Apenas um interpretador. É mais rápido que o interpretador do GHC;
			\begin{itemize}
			 \item \url{http://haskell.org/hugs}
			 \item Pacote hugs no linux
			\end{itemize}
	  \item [GHC] Compilador e interpretador. É o mais usado dos três;
			\begin{itemize}
			 \item \url{http://haskell.org/ghc}
			 \item Pacote ghc no linux
			\end{itemize}
	  \item [NHC] Apenas um compilador. Arquivos produzidos geralmente melhores que o do GHC.
			\begin{itemize}
			 \item \url{http://www.haskell.org/nhc98}
			\end{itemize}
	 \end{description}	 
	\end{frame}
	
\section{Interpretador ghci}

	\begin{frame}{Base}
	 	 \begin{itemize}
	 	  \item Declarações começam com \textit{let}
	 	 \end{itemize}
	 	 \begin{block}{Sintaxe}
	 	  let <identificador> {[}variáveis\dots{]} = <expressão>
	 	 \end{block}
	 	 \begin{block}{Exemplos}
	 	  \begin{itemize}
	 	   \item let a = 12
	 	   \item let soma x y = x+y
	 	  \end{itemize}
	 	 \end{block}
	\end{frame}
	
	\begin{frame}[fragile]{Controle de Fluxo}
		\begin{itemize}
		 \item Geralmente usaremos recursão em vez de laços
		 \item Possui construções do tipo if-then-else
		\end{itemize}
		
		\begin{block}{Exemplo}
		 \begin{lstlisting}
Prelude> if 2>1 then 3 else 9
		 \end{lstlisting}
		\end{block}
	\end{frame}
	
	\section{Operações}
	\begin{frame}{Operadores}
	 \begin{table}[h]
	  \centering
	  \begin{tabular}{c | c}
			operador & operação\\
			\hline
			+ & soma \\
			- & subtração \\
			* & multiplicação\\
			\textasciicircum & potenciação\\
			/ & divisão (real)\\
			. & composição \\
			\$ & aplicação \\
			++ & concatenação de listas\\
			: & inserção em uma lista\\
			!! & acesso a uma posição da lista\\			
	  \end{tabular}
	 \end{table}	 
	\end{frame}
	
	\begin{frame}{Operadores Booleanos}
	 
	 \begin{table}[h]
	  \begin{tabular}{c | c}
	   operador & operação\\
	   \hline
	   \&\& & $\land$ \\
	   || & $\lor$ \\
	   not & $\lnot$ \\
	  \end{tabular}
	 \end{table}
	 
	 \begin{table}[h]
	  \begin{tabular}{c | c}
	   operador & operação\\
	   \hline
	   == & igual \\
	   /= & diferente \\
	   > & maior que \\
	   >= & maior ou igual \\
	   < & menor que \\
	   <= & menor ou igual \\
	  \end{tabular}
	 \end{table} 
	\end{frame}
	
	\section{Cálculo Lambda}

		\begin{frame}{Funções Lambda}
		 
		 \begin{itemize}
		  \item Funções sem nome
		  \item São avaliadas através de substituição de símbolos
		  \item Usa-se redução $\beta$ e $\alpha$ para fazer a avaliação
		 \end{itemize}
		 
		 \begin{block}{Exemplo}
		  \[(\lambda x. \lambda y. x+y)~ 5~ 4\]
		  \[(\lambda y. 5+y)~ 4\]
		  \[5+4\]		  
		 \end{block}		 
		\end{frame}
		
	\begin{frame}[fragile]{Funções Lambda}
	 
	 \begin{block}{Em Haskell}
	  \begin{lstlisting}	
Prelude> (\x -> x*x) 5
Prelude> (\x y ->x+y) 4 5
	  \end{lstlisting}	  
	 \end{block}
	\end{frame}

	\begin{frame}[fragile]{Currying}
		\begin{itemize}
			\item Avaliação parcial de funções
			\item Consequência direta do cálculo lambda
		\end{itemize}
		\begin{block}{Exemplo}
		 \begin{lstlisting}
Prelude> let a = (\x y -> x+y)
Prelude> let b = a 5
		 \end{lstlisting}
		\end{block}
	\end{frame}	
	
	\section{Tipos de Dados}
	
		\begin{frame}[fragile]{Declaração}
			\begin{block}{Listas}
		\begin{lstlisting}
Prelude> [1,2,3,4]
Prelude> []
Prelude> [2,4..16]
Prelude> [1,3..] !! 5
		 \end{lstlisting}
		\end{block}
		
		\begin{block}{Tuplas}
		\begin{lstlisting}
Prelude> ( 1 )
Prelude> (1,2,3)
Prelude> (1,"hello",[1,2])
Prelude> let tuple = (1 , (+) , 2)
		 \end{lstlisting}
		\end{block}
		
		\end{frame}
	
	\subsection{Listas}
	
	\begin{frame}[fragile]{Listas}	 
		\begin{block}{Operações}	  
			\begin{lstlisting}
Prelude> let duplica x = x*x
Prelude> map duplica [1,3..11]
Prelude> map duplica [1,3..] !! 7
Prelude> foldr (+) 0 [1,3,5,7]
Prelude> foldr (-) 0 [1,3,5,7]
Prelude> foldl (-) 0 [1,3,5,7]
Prelude> foldr (-) 2 [1,3,5,7]
Prelude> foldl (-) 2 [1,3,5,7]
Prelude> filter (>3) [1,3,5,7]
			\end{lstlisting}
		\end{block}	 
	\end{frame}
	
	\begin{frame}[fragile]{Folding}
	 \begin{block}{foldr}
	  Associativo a direita
	  \begin{lstlisting}
Prelude> foldr (-) 0 [1,2,3,4]
Prelude> 1 - (2 - (3 - (4 - 0)))
	  \end{lstlisting}
	 \end{block}
	 
	 \begin{block}{foldl}
	  Associativo a esquerda
	  \begin{lstlisting}
Prelude> foldl (-) 0 [1,2,3,4]
Prelude> (((0 - 1) - 2) - 3) - 4
	  \end{lstlisting}
	 \end{block}
	\end{frame}
	
	\begin{frame}{Exercício}
	 \begin{block}{Exercício}
	  Usando fold e map, faça uma função para calcular a média aritmética e outra para o desvio padrão de uma lista de números. 
	 \end{block}
	 
	  \[ \overline{x} = \frac{\sum\limits_{i \in S}i}{\left|S\right|} \]
	  \[ \sigma^2 = \frac{\sum\limits_{i \in S}(i - \overline{x})^2}{\left|S\right|} \]	 

	\end{frame}
	
	\subsection{Determinando Tipos}
		\begin{frame}[fragile]{Determinando Tipos}
		 \begin{block}{Exemplo}
		 \begin{lstlisting}
Prelude> :t 5
 Num a => a
Prelude> :t "Hello"
 "Hello" :: [Char]
Prelude> :t (1,'a')
 Num t => (t, Char)
Prelude> :t (\x y -> x+y)
 Num a => a -> a -> a
		 \end{lstlisting}
		 \end{block}
		\end{frame}
		
	\subsection{Definindo Tipos}
	
		\begin{frame}[fragile]{Data}
		 
		 \begin{itemize}
		  \item Definição de estruturas
		  \item Muito útil com pattern matching
		 \end{itemize}
		 
		 \begin{block}{Definindo}
		  \begin{lstlisting}
Prelude> data Tree a = Leaf 
         | Node (Tree a) a (Tree a)
		  \end{lstlisting}
		 \end{block}		 
		 \begin{block}{Construindo}
		  \begin{lstlisting}
Prelude> let n2 = Node Leaf 10 Leaf
Prelude> let n1 = Node Leaf 6 Leaf
Prelude> let n0 = Node n1 7 n2
		  \end{lstlisting}
		 \end{block}
		\end{frame}
		
		\subsection{Pattern Matching}
	
		\begin{frame}[fragile]{Pattern Matching}	 
		 \begin{block}{Exemplo}
		  \begin{lstlisting}
Prelude> let a = (\(x,y) -> x+y)
Prelude> let b = (\(x:y:_) -> x*y)
Prelude> let (p,k) = (\x y -> (x,y)) 3 4
Prelude> let Node esq valor dir = n0
		  \end{lstlisting}
		 \end{block}
		\end{frame}
	
	\section{IO}
	
		\begin{frame}[fragile]{Funções de Saída}
		 \begin{itemize}
		  \item Não são funções de fato
		  \item Permitem efeitos colaterais
		  \item São consideradas ``IO Actions''
		  \item Sempre retornam valores de um tipo IO
		  \item Usa-se o comando \textit{do} para executar uma sequência de ações
		 \end{itemize}
		 
		 \begin{block}{Print}
		  \begin{lstlisting}
print :: Show a => a -> IO ()
putStr :: String -> IO ()
putStrLn :: String -> IO ()
		  \end{lstlisting}
		 \end{block}
		\end{frame}
		
		\begin{frame}[fragile]{Funções de Saída}
		 \begin{block}{printing}
		  \begin{lstlisting}
Prelude> putStrLn "Nova linha"
Prelude> putStr "Mesma linha"
Prelude> print 42
		  \end{lstlisting}
		 \end{block}
		\end{frame}
		
		\begin{frame}[fragile]{Funções de Entrada}

		 \begin{block}{Definição}
		  \begin{lstlisting}
readFile :: FilePath -> IO String
getLine :: IO String
		  \end{lstlisting}
		 \end{block}
		 
		 \begin{block}{}
		  \begin{lstlisting}
Prelude> file <- readFile "arquivo"
Prelude> line <- getLine
		  \end{lstlisting}
		 \end{block}		 
		\end{frame}

		\subsection{Conversões}

			\begin{frame}[fragile]{String}
			 \begin{block}{Funções}
			  \begin{lstlisting}
Prelude> show 32
Prelude> read "32"::Int
Prelude> read "32"::Float
			  \end{lstlisting}
			 \end{block}
			\end{frame}
			
			\begin{frame}[fragile]{Números}
			 \begin{block}{Funções}
				\begin{lstlisting}
Prelude> :t fromIntegral
fromIntegral :: (Integral a, Num b) => a -> b
Prelude> :t toRational
Real a => a -> Rational
				\end{lstlisting}
			 \end{block}				
			\end{frame}
			
			\begin{frame}[fragile]{Aproximações}
			 \begin{block}{Funções}
			  \begin{lstlisting}
Prelude> :t ceiling
ceiling :: (Integral b, RealFrac a) => a -> b
Prelude> ceiling (-3.2)
Prelude> floor (-3.2)
Prelude> truncate (-3.2)
Prelude> round (-3.2)
			  \end{lstlisting}
			 \end{block}
			\end{frame}
			
			\begin{frame}{Aproximações}
			 \[ a \leq ceiling~ a\]
			 \[ floor~ a \leq a\]
			 \[ \left|truncate~ a\right| \leq \left|a\right|\]
			\end{frame}

	\section{Arquivos}
	
		\begin{frame}[fragile]{Arquivos}
		 \begin{itemize}
		  \item Código pode ser dividido em diversos arquivos
		  \item Permite a criação de módulos para facilitar reúso
		  \item Usa-se a extensão .hs
		 \end{itemize}
		 
		 \begin{block}{Exemplo}
		  \begin{lstlisting}
Module Main where

main = do
     putStrLn "Hello World"
		  \end{lstlisting}
		 \end{block}
		\end{frame}
		
		\begin{frame}[fragile]{Definição de Funções}
		 \begin{block}{Formato}
		  \begin{lstlisting}
<identificador> [variáveis...] = <expressão>
		  \end{lstlisting}
		 \end{block}
		 
		 \begin{block}{Exemplo}
		  \begin{lstlisting}
fatorial 0 = 1
fatorial 1 = 1
fatorial n = n * fatorial (n-1)
		  \end{lstlisting}
		 \end{block}
		\end{frame}
		
		\begin{frame}[fragile]{Let e Where}
		 \begin{itemize}
		  \item Permite a definição de variáveis auxiliares
		  \item Restringem o escopo
		  \item Múltiplas definições devem estar alinhadas
		 \end{itemize}
		 \begin{block}{Let-in}
		  \begin{lstlisting}
variance l =
 let avg = average l
     dif = map (\x -> (x - avg)^2) l
 in average dif
		  \end{lstlisting}	  
		 \end{block}
		\end{frame}
		
		\begin{frame}[fragile]{Where}
		 \begin{block}{Where}
		  \begin{lstlisting}
variance l =
 average dif
 where avg = average l
       dif = map (\x -> (x - avg)^2) l
		  \end{lstlisting}
		 \end{block}		 
		\end{frame}
		
		\begin{frame}{Exercício}
		 
		\end{frame}
		
		\subsection{Módulos}
			\begin{frame}[fragile]{Módulos}
			 \begin{itemize}
			  \item Servem para organizar o código
			  \item Permitem reúso de código
			  \item Fáceis de se usar
			 \end{itemize}
			\end{frame}
			
			\begin{frame}[fragile]{Definição}
			 \begin{block}{Exemplo}
			  \lstinputlisting{exemplos/modules/Arvore.hs}
			 \end{block}
			\end{frame}
			
			\begin{frame}[fragile]{Import}
			 \begin{block}{Variações}
			  \begin{description}
			   \item [import Module] traz todas as funções de Module
					\begin{itemize}
					 \item x, y, z, Module.x, Module.y, Module.z
					\end{itemize}
				 \item [import Module (x,y)] traz somente x e y
				 \item [import Module as Foo] trata Module como Foo
					\begin{itemize}
					 \item x, y, z, Foo.x, Foo.y, Foo.z
					\end{itemize}

			   \item [import qualified Module] requer o nome do módulo prefixado
					\begin{itemize}
					 \item Module.x, Module.y, Module.z
					\end{itemize}
			  \end{description}			  
			 \end{block}
			\end{frame}
			
			\begin{frame}[fragile]{Import}
			 \begin{block}{Exemplo}
			  \lstinputlisting{exemplos/src/arvore.hs}
			 \end{block}
			\end{frame}
			
\section{Referências}
\begin{frame}{Referências}
	\begin{thebibliography}{9}
	\bibitem{YAHT} Hal Daumé III, \textit{Yet Another Haskell Tutorial}
	\bibitem{THSE} Hudak, Paul. \textit{The Haskell School of Expression}
	\end{thebibliography}
\end{frame}
	
	
\end{document}
