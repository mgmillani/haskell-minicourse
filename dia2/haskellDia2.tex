\documentclass{beamer}
\usepackage[T1]{fontenc}
\usepackage[utf8]{inputenc}
\usepackage[brazilian]{babel}
\usepackage{hyperref}
\usepackage{listingsutf8}
\usepackage{color}
\usepackage{theoremref}
\usepackage{verbatim,amssymb,latexsym,amsmath,mathrsfs}

\usecolortheme{rose}
\usetheme{Hannover}

\hypersetup
{
  colorlinks=true,
  urlcolor=blue,
  citecolor=blue,
  linkcolor=blue
}

\definecolor{darkGreen}{rgb}{0,0.6,0}

\lstset{
	extendedchars=true,
	inputencoding=utf8,
	basicstyle=\small,
	keywordstyle=\color{red}\bfseries,
	identifierstyle=\color{darkGreen},
	commentstyle=\color{gray}\textit,
	stringstyle=\color{magenta},
	showstringspaces=false,
	tabsize=1,
	language=Haskell,
	numbers=left,         % Números na margem da esquerda
	numberstyle=\tiny,    % Números pequenos
	stepnumber=1,         % De 1 em 1
	numbersep=5pt         % Separados por 5pt
}

\lstset
{
	literate={ê}{{\^e}}1 {ç}{{\c{c}}}1 {ã}{{\~a}}1 {â}{{\^a}}1 {õ}{{\~o}}1 {í}{{\'i}}1 {é}{{\'e}}1 {á}{{\'a}}1 {ó}{{\'o}}1 {ú}{{\'u}}1
}

\title{Mini-Curso de Haskell - SEMAC 2013/1}
\author[M. Millani]{Marcelo Garlet Millani}
\institute[II/UFRGS]{Universidade Federal do Rio Grande do Sul \\ Instituto de Informática \\ Grupo PET Computação}

\begin{document}

	\begin{frame}
		\maketitle
	\end{frame}

	\section{Básico de OpenGl}
		\begin{frame}{}
		 \tableofcontents[currentsection,currentsubsection]
		\end{frame}
		
		\begin{frame}[fragile]{OpenGl}
		 \begin{itemize}
		  \item OpenGl é uma biblioteca gráfica para C
		  \item Permite aproveitar a GPU para desenhos
		  \item Capaz de trabalhar em 3D, mas usaremos apenas 2D
		  \item GHC já vem com bindings para Haskell
		  \item Módulo Graphics.Rendering.OpenGL
				\begin{itemize}
				 \item Pacote libghc-opengl-dev no Linux (Debian)
				\end{itemize}
		  \item Usaremos a GLUT (Graphics.UI.GLUT) para a janela
				\begin{itemize}
				 \item Pacote libghc-glut-dev no Linux (Debian)
				\end{itemize}
		 \end{itemize}
		\end{frame}
		
		\begin{frame}[fragile]{Inicialização}
		 \begin{block}{Exemplo}
			\begin{lstlisting}
main = do 
  (progname, _) <- getArgsAndInitialize
  createWindow "Hello World"
  displayCallback $= display myPoints
  mainLoop
  
display p = do 
  clear [ColorBuffer]
  renderPrimitive LineLoop $ do
    mapM_ (\(x,y,z)->vertex$Vertex3 x y z) p
  flush
			\end{lstlisting}
		 \end{block}

		\end{frame}


	\section{Qualified Types}
		\begin{frame}{}
		 \tableofcontents[currentsection,currentsubsection]
		\end{frame}
		
		\begin{frame}[fragile]{Para que servem}
			\begin{itemize}
			  \item Uso de funções polimórficas
				\item Restringem os tipos para os quais a função está definida
			\end{itemize}
			
			\begin{block}{Exemplo}
			 Função de soma:
			 \begin{lstlisting}
(+) :: Num a=> a->a->a
			 \end{lstlisting}
			 Definida para qualquer $a$ tal que $a$ seja um membro de $Num$.
			\end{block}
		 
		\end{frame}
		
		\begin{frame}[fragile]{Como usar}
		 
		 \begin{itemize}
		  \item Definimos como as funções da classe serão aplicadas para um certo tipo de dado
		  \item Em alguns casos, não é necessário definir todas as funções
		  \begin{itemize}
		   \item Por exemplo, basta definir igualdade (==) para se ter diferença (/=)
		  \end{itemize}

		 \end{itemize}
		 
		 \begin{block}{Exemplo}
		  \begin{lstlisting}
instance Eq a => Eq (Figure a) where
	Circle r == Circle s           = r == s
	Rectangle w h == Rectangle a b = w == a && h == b
		  \end{lstlisting}
		 \end{block}		 
		\end{frame}
		
		\begin{frame}[fragile]{Definindo Classes}
		 
		 \begin{block}{Exemplo}
class Drawable a where
	vertexes :: a -> [(GLfloat,GLfloat,GLfloat)]
	draw :: a ->
		 \end{block}

		\end{frame}

			
\section{Referências}
\begin{frame}{Referências}
	\begin{thebibliography}{9}
	\bibitem{YAHT} Hal Daumé III, \textit{Yet Another Haskell Tutorial}
	\bibitem{THSE} Hudak, Paul. \textit{The Haskell School of Expression}
	\bibitem{HAGL} \url{http://www.haskell.org/haskellwiki/OpenGLTutorial1}
	\end{thebibliography}
\end{frame}
	
	
\end{document}
